\section{Distribuições de Probabilidade}

Denomina-se de distribuição de probabilidade de alguma variável aleatória a regra geral que define a função de massa de probabilidade (variável discreta) ou de densidade de probabilidade (variável contínua) para a variável de interesse.

\subsection{Distribuições Discretas}

\subsubsection{Distribuição Uniforme Discreta Ud(a,b)}

\begin{equation}
    p(x) = \frac{1}{b-a+1}
\end{equation}

\subsubsection{Distribuição Binomial Bin(n,$\theta$)}

\begin{equation}
    p(x) = \begin{pmatrix} n\\x \end{pmatrix} \cdot \theta^x \cdot (1-\theta)^{n-x}
\end{equation}

\subsubsection{Distribuição Hipergeométrica Hip(M, N, n)}

\begin{equation}
    p(x) = \frac{\begin{pmatrix} M\\x \end{pmatrix} \cdot \begin{pmatrix} N-M\\n-x \end{pmatrix}}{\begin{pmatrix} N\\n \end{pmatrix}}
\end{equation}

\subsubsection{Distribuição de Poisson Poi($\mu$)}

\begin{equation}
    p(x) = \frac{e^\mu \mu^x}{x!}
\end{equation}

\subsubsection{Distribuição Binomial Negativa BinN(a,$\theta$)}

\begin{equation}
    p(x) = \frac{(a + x - 1)!}{x! (a-1)!} \cdot \theta^a(1-\theta)^x
\end{equation}

\subsection{Distribuição Contínua}

\subsubsection{Distribuição Uniforme U(c,d)}

\begin{equation}
    p(x) = \frac{1}{d-c}
\end{equation}

\subsubsection{Distribuição Beta($\alpha$, $\beta$)}

\begin{equation}
    p(x) = \frac{\Gamma(\alpha+\beta)}{\Gamma(\alpha) \Gamma(\beta)} \cdot x^{\alpha+1} (1-x)^{\beta-1}
\end{equation}

\subsubsection{Distribuição Beta Gerenalizada}

\subsubsection{Distribuição Exponencial}

\subsubsection{Distribuição Gama}

\subsubsection{Distribuição Gama Inversa}

\subsubsection{Distribuição Normal}

\subsubsection{Distribuição Student}

\subsection{Exemplo de códigos gerados no R}

\lstinputlisting[language=Octave]{code/graficos.r}
