\section{Variáveis aleatórias}

As variáveis aleatórias são de dois tipos. As discretas associadas a alguma contagem, e as contínuas que envolvem medições ou mesmo razões. A distinção é necessária para que as regras do cálculo das probabilidades sejam corretamente aplicadas a cada caso. Segue um resumo das propriedades e distinções dos principais modelos de probabilidades para variáveis aleatórias discretas e contínuas.

\subsection{Variáveis Aleatórias Discretas}

Uma variável aleatória discreta $X$ é uma função que associa as proposições de interesse a um conjunto enumerável (ou categórico, que pode ser nominal ou ordinal), não necessariamente finito, de valores. Se, por exemplo, $X$ caracteriza o número de amostras de sangue avaliadas antes do aparecimento de uma amostra que contém um vírus de interesse, então os possíveis valores para $X$ que caracterizam um conjunto infinito 0, 1, 2, ... já que náo existe um limite superior para delimitá-lo.

De modo geral, o valor $Pr(X = x_j | H)$ denota a massa de probabilidade no ponto $x_j$. Sobre o conjunto de todos os pontos plausíveis, ou seja, pontos com massa de probabilidade maior que zero, a Lei da adição garante que:

\begin{equation}
    \sum_{j}{Pr(X = x_j|H)} = 1
\end{equation}

A média ou valor esperado de $X$, $E(X)$, e a sua variância, $V(X)$, são definidas por:

\begin{equation}
\begin{split}
    E(X) = \sum_{j}{x_j Pr(X = x_j|H)} \\
    V(X) = \sum_{j}{(x_j - E(X))^2 Pr(X = x_j|H)}
\end{split}
\end{equation}

\subsection{Variáveis Aleatórias Contínuas}

Quando os possíveis valores de $X$ foram um subconjunto que compreende pelo menos um intervalo da escala de números reais, a variáveis aleatória é denominada contínua. Variáveis aleatórias contínuas são caracterizadas pela sua função distribuição cumulativa $F(x)$ que é definida para todo o número real $x \in (-\infty \infty)$.

\begin{equation}
    F(x) = Pr(X \leq x | H)
\end{equation}

A função densidade de probabilidade é definida por $f(x) = \frac{F(x)}{dx}$. A integração da densidade de probabilidade sobre os números reais é igual a 1, ou seja: $\int_{-\infty}^{\infty}{f(x) \cdot dx} - 1$.

A média e a variância de $X$ também são expressos em integrais:

\begin{equation}
\begin{split}
    E(X) = \int{x \cdot f(x) dx} \\
    V(X) = \int{(x - E(X))^2 \cdot f(x) dx}
\end{split}
\end{equation}