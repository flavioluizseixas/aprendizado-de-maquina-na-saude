\section{Paradigmas Frequentista e Bayesiano}

O paradigma frequentista admite a probabilidade num contexto restrito a fenômenos que podem ser medidos por frequências relativas. O paradigma Bayesiano entende-se que a probabilidade é uma medida racional e condicional de incerteza. Uma medida do grau de plausibilidade de proposições quaisquer, as quais não precisam necessariamente estar associadas a fenômenos medidos por frequência relativa. Por exemplo, no paradigma Bayesiano admite-se falar da probabilidade de extinção de uma espécie, o que não seria admissível sob o paradigma frequentista.

A inferência estatística é o processo formal utilizado para fazer afirmações genéricas com base em informações parciais. Essas afirmações sáo probabilísticas pois se caracterizam por incluir componentes de incerteza.

Na perspectiva bayesiana, a inferência estatística sobre qualquer quantidade de interesse é descrita como a modificação que se processa nas incertezas à luz de novas evidências.

A conceituação frequentista admite falar em probabilidades somente no contexto de frequências relativas. Em contraste, na conceituação bayesiana, probabilidades quantificam as plausibilidades de proposições ou eventos. Ao atribuir plausibilidades diferenciadas a proposições, a formalização bayesiana de probabilidade estende a lógica dedutiva, restrita a classificar proposições em verdadeiras (probabilidade igual a 1) ou falsas (probabilidade igual a zero), para um conjunto de possibilidades entre estes dois extremos.

O rápido crescimento do uso do paradigma bayesiano em ciências aplicadas ao longo das últimas décadas foi facilitado pelo surgimento de vários programas para efetuar as computações estatísticas. Entre esses, destaca-se o R (programa de livre distribuições e de código aberto).
