\section{O Teorema de Bayes}

Mutas propriedades do cálculo de probabilidades podem ser deduzidas a partir das três leis básicas indicadas na seção anterior. Depois teoremas adicionais merecem especial destaque, o Teorema da Probabilidade Total e o Teorema de Bayes.

\begin{tcolorbox}[colback=blue!5,colframe=blue!75!black,title=Teorema da Probabilidade Total]
    Seja ${E_1; j=1, ..., m}$ um conjunto de $m$ eventos exclusivos e exaustivos sob $H$, e seja $A$ outro evento qualquer. Então $Pr(A|H)$ pode ser reescrito estendendo a conversa para a inclusão dos $E_j$.
    \begin{equation}
        Pr(A|H) = \sum_{j=1}^{m}{Pr(A|E_j H) \cdot Pr(E_j|H)}
    \end{equation}
\end{tcolorbox}

\begin{tcolorbox}[colback=blue!5,colframe=blue!75!black,title=Teorema de Bayes]
    Sejam $E$ e $F$ dois eventos quaisquer e $Pr(E|H)>0$, então:
    \begin{equation}
        Pr(F|E H) = \frac{Pr(E|F H) \cdot Pr(F|H)}{Pr(E|H)}
    \end{equation}
\end{tcolorbox}

\subsection{Exemplo}

Um estudo de uma mamografia no diagnóstico de câncer é apresentado na Tabela~\ref{tab:tab1}. Os dados foram obtidos experimentalmente sobre a efetividade do exame na detecção de um tumor de mama maligno ou benigno. Por exemplo, se um tumor é maligno $Ca$, a probabilidade de que o exame resulte positivo é $Pr(Pos | Ca) = 0,792$, ou seja, 79,2\%. De forma similar temos $Pr(Neg | Ca') = 0,904$ como a probabilidade de que o exame resulte negativo se o tumor não é maligno $(Ca')$. Os percentuais para faltos positivos e falsos negativos são 9,6\% e 20,8\%, respectivamente.

\begin{table}[h]
    \caption{Resultados dos testes de câncer de mamas}
    \label{tab:tab1}
    \center
    \begin{tabular}{ |c|c|c| } 
    \hline
    \multirow{2}{4em}{Resultado do teste} & \multicolumn{2}{|c|}{Realidade} \\
    & $Ca$ (Tumor maligno) & $Ca'$ (Tumor benigno) \\ 
    \hline
    $Pos$ (Positivo) & 0,792 & 0,096 \\ 
    $Neg$ (Negativo) & 0,208 & 0,904 \\ 
    \hline
    \end{tabular}
\end{table}

Com essa tabela, fez-se a seguinte pergunta: "Suponha que uma paciente pertença a uma população (mesmo grupo etário, hábito alimentar, etc.) na qual a incidência geral de câncer de mama é de 1\%. Detectado um nódulo no seio desta paciente, pede-se uma mamografia para avaliar a possibilidade de que se trate de um tumor maligno; o resultado é positivo. De posse deste conjunto de informações, qual é, em sua opinião, a probabilidade de tratar-se de um tumor maligno?"

Pelo Teorema de Bayes:
\begin{equation}
\begin{split}
Pr(Ca|Pos) & = \frac{Pr(Pos|Ca) \cdot Pr(Ca)}{Pr(Pos)} \\
& = \frac{Pr(Pos|Ca) \cdot Pr(Ca)}{P(Pos|Ca) \cdot P(Ca) + Pr(Pos|Ca') \cdot Pr(P(Ca')} \\
& = \frac{0,792 \cdot 0,01}{0,792 \cdot 0,01 + 0,096 \cdot 0,99} = 0,077
\end{split}
\end{equation}

Observe que a acurácia retrospectiva do exame $Pr(Pos|Ca)$ é diferente de acurácia preditiva $Pr(Ca|Pos)$. Na prática, a importância atribuída à alta probabilidade de um resultado positivo quando o tumor é de fato maligno, $Pr(Pos|Ca) = 0,792$ foi excessiva com relação a baixa probabilidade de incidência desse tipo de câncer $Pr(Ca)=0,01$. No contexto mais geral de investigação científica, o Teorema de Bayes expressa o mecanismo pelo qual hipóteses científicas e evidências empíricas são integrados.

Seja $F$ uma hipótese científica cuja probabilidade corrente é $Pr(F|H)$. Seja $E$ a evidência contida nos dados, experimentais ou observacionais, e cuja probabilidade sob a premissa $F$ é dada por $Pr(E|F,H)$. Então, se efetivamente foi observado $E$, o Teorema de Bayes permite calcular a probabilidade atualizada $Pr(F|E,H)$ da hipótese $F$. Portanto, a probabilidade atualizada de $F$ é composta pela sua probabilidade a priori, modificada pela acresção das novas evidências $E$ presentes nos dados.