\section{Análise Bayesiana de Dados}

A inferência bayesiana consiste na construção de uma distribuição de probabilidade posterior via o Teorema de Bayes. Essa distribuição resulta da combinação de informações prévias, sumarizadas em uma distribuição denominada priori, com dados estatísticos descritos por algum modelo probabilístico e resumidos na função de verossimilhança.

A distribuição posterior é a forma mais completa de expressar o estado do conhecimento sobre o fenômeno investigado. Toda pergunta específica é respondida a partir da análise da distribuição posterior. Ela contém toda a informação necessária para a inferência.

Além disso, o processo é dinâmico. A distribuição posterior de hoje pode se transformar na priori em estudos futuros, caracterizando o elemento cumulativo de aquisição de informações.

